\section{Inductive Logic Programming}
\veronika{not sure if the title should be rather 'mining logical rules' or 'learning' - wo can encompass/evaluate all kinds of approaches, statistical, neural, ... maybe talk about ILP but mention in intro that we do not have restrictions there.}
% \veronika{see my slack notes: I would\\
% 1 describe ILP neural ILP and existing approaches in intro (existing does not have to be that detailed but short overview - is not directly related work for dataset generation, no? we can describe the ones we evaluate in some more detail maybe with experiments?)\\
% 2 then introduce inference with Evans example\\
% - put theory necessary for evaluator in evaluator section and leave rule explanation in dataset section? not sure about the latter\\
% 3 then describe datasets as related work
% - existing evaluation approaches are also related work but I am not sure where to describe this best\\
% or make very long intro including 1 \& 2 (maybe as subsections) and make extra section related work as section 2?
% }
\subsection{Example}

%motivating example mentioning categories to show capabilities of the ILP system:


% \cristina{describe logic basics, inference, datalog and other types of logic (for further extensions)}
% \veronika{since we only mention extensions briefly, as future work, I would not mention them here. the background is supposed to describe the background of what we are doing here I would say...}
\cristina{describe classic ILP task (predicate invention etc) 
%and classic frameworks and the one we will use to compare: FOIL, ProGol)
%Veronika - put that into experiments
}

\subsection{Evaluation}%Related Work

\veronika{existing datasets and measures }